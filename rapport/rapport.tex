% Edgar Fournival <contact@edgar-fournival.fr>

\documentclass[12pt,a4paper]{article}
\usepackage[left=2.5cm,right=2.5cm,top=2.5cm,bottom=2.5cm]{geometry}

\usepackage[francais]{babel}

\usepackage{ifxetex}

\ifxetex
	\usepackage{fontspec}
\else
	\usepackage[T1]{fontenc}
	\usepackage[utf8]{inputenc}
\fi

\usepackage{amsmath}
\usepackage{amssymb}
\usepackage{eurosym}
\usepackage{mathtools}

\usepackage[noend]{algorithmic}
\renewcommand{\algorithmicif}{\textbf{si}}
\renewcommand{\algorithmicthen}{\textbf{alors}}
\renewcommand{\algorithmicfor}{\textbf{pour}}
\renewcommand{\algorithmicdo}{\textbf{faire}}

\usepackage{enumitem}
\usepackage{adjustbox}
\usepackage[hidelinks]{hyperref}

\setlength\parindent{0pt}
\setlength\parskip{0.25em}

\let\emptyset\varnothing
\let\leq\leqslant
\let\geq\geqslant

\begin{document}

\section*{Introduction}

\section{Mission}

\subsection{Problème posé}

L'exploration de l'arbre d'un semigroupe numérique n'est pas triviale et est apparentée au problème de Frobenius.

Le mathématicien Georg Frobenius s'est posé la question suivante : quel est le montant maximal que l'on peut pas rendre en fonction de pièces de monnaie données ?

Ce montant se nomme ``nombre de Frobenius'' et est plus formellement décrit par le plus grand entier qu'il est impossible de calculer à partir de coefficients donnés.

On peut aussi visualiser ce problème en considérant les scores au Rugby : quels sont les scores que l'on ne peut pas obtenir avec le but (3 points), l'essai (5 points) et l'essai transformé (7 points) ?

La résolution des scores au Rugby est simple, on ne peut pas avoir 1, 2 ou 4 points; tous les autres scores sont possibles. En effet, il est aisé de calculer le nombre de Frobenius (ici, 4) pour $n \leq 3$ où $n$ est le nombre d'entiers possibles pour la combinaison. Ici, $n = 3$ soit $\{3,5,7\}$.

\section{Déroulement}

\subsection{Jeudi 12 janvier 2017}

Installation au LRI, explications :
\begin{itemize}
	\item	parcours de l'arbre de semi-groupes numériques

	\item	structures de données utilisées, notamment dans \texttt{NumericMonoid}

	\item	algorithme ``naïf'' : tableau de booléens (le nombre est-il dans le semi-groupe ?) de taille $B =$ genre, soit la profondeur dans l'arbre et le nombre de trous
			\begin{algorithmic}
				\FOR {$g \text{ de dernierEnlevé à } B$}
					\IF{$T[g]$}
						\FOR {$i \text{ de } 1 \text{ à } \lfloor \frac{g-1}{2} \rfloor$}
							\IF{$T[i]$ et $T[g-1]$}
								\STATE $g$ est générateur
							\ENDIF
						\ENDFOR
					\ENDIF
				\ENDFOR
			\end{algorithmic}

	\item	algorithme optimisé facilitant la vectorisation : tableau d'entiers modélisant le nombre de paires ($i, j \in$ SN tel que $i + j = g$ et $i \leq j$), tous les nombres avec exactement une paire sont des générateurs
\end{itemize}

\section{Résultats}

\section*{Conclusion}

\end{document}
